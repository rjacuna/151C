\documentclass{article}
\usepackage{fontspec}

% Used to embed Sage code in latex
%\usepackage{sagetex}

% Math Environment
\usepackage{euler}        % Euler font
\usepackage{amsmath}      % Math macros
\usepackage{amssymb}      % Math symbols
\usepackage{unicode-math} % Unicode support


\usepackage[makeroom]{cancel} % Used to cancel terms in algebraic equations
\usepackage{ulem} % Different underline environments
\usepackage{polynom} %Polynomial long division

% Typesetting Rules
\setlength\parindent{0em}
\setlength\parskip{0.618em}
\usepackage[a4paper,lmargin=1in,rmargin=1in,tmargin=1in,bmargin=1in]{geometry}
\setmainfont[Mapping=tex-text]{Helvetica Neue LT Std 45 Light}

% Common Macros
\newcommand\N{\mathbb{N}}
\newcommand\Z{\mathbb{Z}}
\newcommand\Q{\mathbb{Q}}
\newcommand\R{\mathbb{R}}
\newcommand\C{\mathbb{C}}
\newcommand\A{\mathbb{A}}
\def\res{\mathop{\text{Res}}\limits}

% Color
\usepackage[dvipsnames]{xcolor}
\usepackage{pagecolor}
% \definecolor{DeepMossGreen}{HTML}{394820}
% \pagecolor{DeepMossGreen}
% \color{Goldenrod}

\begin{document}
Suppose $f$ is a continuous function on $R^1$,$f(x+2\pi) = f(x)$, and
$\alpha/\pi$ is irrational.

Prove that
\[\lim_{N\rightarrow \infty} \frac{1}{N}\sum_{n=1}^N f(x+n\alpha)
  =\frac{1}{2\pi}\int_{-\pi}^\pi f(t) dt\]
for every $x$. \textit{Hint:}  do it first for $f(x) = e^{ikx}$.

\uwave{pf.}

$f$ is continuous on $\R \implies f \in \mathcal{R}$, furthermore $f(x+2\pi) = f(x)$ so the period of $f$ is $2\pi$.

Since $\alpha/\pi$ is irrational $\not\exists k\in \Z:\, n\alpha  =
k\pi\quad (n = 1,2,\dots, N).$ So $f(x+n\alpha) \neq f(x+k\pi)$.

Furthermore $\{f(x+n\alpha)\}_{n=1}^N \subset f([-\pi,\pi])$, because
$f(-\pi)= f(\pi)$ and $f$ is continuous with period $2\pi$.

Furthermore, $f(x+n\alpha)$
are distinct points in  $f([-\pi,\pi])$.

Since, $f \in \mathcal{R}$ it
follows that a fine enough partition by sample points is equal to the
integral.

Let $P$ be the evenly
distributed partition of $[-\pi,\pi]$. Then $\varDelta x_n =
\frac{\pi-(-\pi)}{N} = \frac{2\pi}{N}$.

\[\int_{-\pi}^\pi f(x) dx = \lim_{n\rightarrow \infty} \sum_{n=1}^N
  f(x +n\alpha)\frac{2\pi}{N} \implies \lim_{N\rightarrow \infty} \frac{1}{N}\sum_{n=1}^N f(x+n\alpha)
  =\frac{1}{2\pi}\int_{-\pi}^\pi f(t) dt\quad \blacksquare\]

\end{document}



%%% Local Variables:
%%% mode: latex
%%% TeX-master: t
%%% End:
