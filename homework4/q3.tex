\documentclass{article}
\usepackage{fontspec}

% Used to embed Sage code in latex
%\usepackage{sagetex}

% Math Environment
\usepackage{euler}        % Euler font
\usepackage{amsmath}      % Math macros
\usepackage{amssymb}      % Math symbols
\usepackage{unicode-math} % Unicode support


\usepackage[makeroom]{cancel} % Used to cancel terms in algebraic equations
\usepackage{ulem} % Different underline environments
\usepackage{polynom} %Polynomial long division

% Typesetting Rules
\setlength\parindent{0em}
\setlength\parskip{0.618em}
\usepackage[a4paper,lmargin=1in,rmargin=1in,tmargin=1in,bmargin=1in]{geometry}
\setmainfont[Mapping=tex-text]{Helvetica Neue LT Std 45 Light}

% Common Macros
\newcommand\N{\mathbb{N}}
\newcommand\Z{\mathbb{Z}}
\newcommand\Q{\mathbb{Q}}
\newcommand\R{\mathbb{R}}
\newcommand\C{\mathbb{C}}
\newcommand\A{\mathbb{A}}
\def\res{\mathop{\text{Res}}\limits}

% Color
\usepackage[dvipsnames]{xcolor}
\usepackage{pagecolor}
% \definecolor{DeepMossGreen}{HTML}{394820}
% \pagecolor{DeepMossGreen}
% \color{Goldenrod}

\begin{document}
If $\alpha$ is real and $-1 <x < 1$, prove Newton's binomial theorem,
\[(1+x)^{\alpha} = 1+\sum_{n=1}^\infty \frac{\alpha(\alpha
    -1)\cdots(\alpha-n+1)}{n!} x^n\]

\uwave{pf.}

Let $f(x) = 1+\sum_{n=1}^\infty \frac{\alpha(\alpha
  -1)\cdots(\alpha-n+1)}{n!} x^n$, by the quotient test we have,
\begin{align*}
  R&= \lim_{n \rightarrow \infty} \frac{\left|  \frac{\alpha(\alpha
      -1)\cdots(\alpha-n+1)}{n!}\right|}{\left|\frac{\alpha(\alpha
     -1)\cdots(\alpha-(n+1)+1)}{(n+1)!}\right|}\\
   &= \lim_{n \rightarrow \infty} \left|\frac{\alpha(\alpha
      -1)\cdots(\alpha -(n-1)+1)(\alpha-n+1)(n+1)n!}{\alpha(\alpha
  -1)\cdots(\alpha -(n-1))(\alpha-n)n!}\right|\\
   &= \lim_{n \rightarrow \infty} \left|\frac{\alpha(\alpha
      -1)\cdots(\alpha-n+1)(n+1)}{\alpha(\alpha
     -1)\cdots(\alpha -n +1)(\alpha-n)}\right|\\
  &= \lim_{n \rightarrow \infty} \frac{n+1}{\left|\alpha-n\right|} = 1\\
\end{align*}
So $f$ converges for $|x|<1$

\begin{align*}
  f'(x) &= \sum_{n=1}^\infty \frac{\alpha(\alpha
  -1)\cdots(\alpha-n+1)}{(n-1)!} x^{n-1}\\
  &\implies f'(x) + xf'(x) = \sum_{n=1}^\infty \frac{\alpha(\alpha
  -1)\cdots(\alpha-n+1)}{(n-1)!} x^{n-1} + x\sum_{n=1}^\infty \frac{\alpha(\alpha
    -1)\cdots(\alpha-n+1)}{(n-1)!} x^{n-1}\\
  &\implies (1+x)f'(x) = \sum_{n=1}^\infty \frac{\alpha(\alpha
  -1)\cdots(\alpha-n+1)}{(n-1)!} x^{n-1} + \sum_{n=1}^\infty \frac{\alpha(\alpha
    -1)\cdots(\alpha-n+1)}{(n-1)!} x^{n}\\
  k = n-1 &\implies (1+x)f'(x) = \sum_{k=0}^\infty \frac{\alpha(\alpha
            -1)\cdots(\alpha-k)}{k!} x^{k} + \sum_{n=1}^\infty \frac{\alpha(\alpha
            -1)\cdots(\alpha-n+1)}{(n-1)!} x^{n}\\
          &\implies (1+x)f'(x) = \alpha + \sum_{k=1}^\infty \frac{\alpha(\alpha
            -1)\cdots(\alpha-k)}{k!} x^{k} + \sum_{n=1}^\infty \frac{\alpha(\alpha
            -1)\cdots(\alpha-n+1)}{(n-1)!} x^{n}\\
  &\implies (1+x)f'(x) = \alpha + \sum_{n=1}^\infty \frac{\alpha(\alpha
            -1)\cdots(\alpha-n)}{n!} + \frac{\alpha(\alpha
    -1)\cdots(\alpha-n+1)}{(n-1)!} x^{n}\\
  &\implies (1+x)f'(x) = \alpha +\alpha \sum_{n=1}^\infty \frac{(\alpha
            -1)\cdots(\alpha-n)}{n!} + \frac{(\alpha
    -1)\cdots(\alpha-n+1)}{(n-1)!} x^{n}\\
  &\implies (1+x)f'(x) = \alpha +\alpha \sum_{n=1}^\infty \frac{(\alpha
            -1)\cdots(\alpha-n) + n(\alpha
    -1)\cdots(\alpha-n+1)}{n!} x^{n}\\
  &\implies (1+x)f'(x) = \alpha +\alpha \sum_{n=1}^\infty \frac{\color{blue}(\alpha
            -1)\cdots(\alpha -n+1)\color{black}(\alpha-n) + n\color{blue}(\alpha
    -1)\cdots(\alpha-n+1)\color{black}}{n!} x^{n}\\
  &\implies (1+x)f'(x) = \alpha +\alpha \sum_{n=1}^\infty \frac{\color{blue}(\alpha
    -1)\cdots(\alpha -n+1)\color{black}[(\alpha-n) + n]}{n!} x^{n}\\
  &\implies (1+x)f'(x) = \alpha +\alpha \sum_{n=1}^\infty \frac{\alpha(\alpha
    -1)\cdots(\alpha -n+1)}{n!} x^{n}\\
  &\implies (1+x)f'(x) = \alpha\left(1 + \sum_{n=1}^\infty \frac{\alpha(\alpha
    -1)\cdots(\alpha -n+1)}{n!} x^{n}\right)\\
  &\implies (1+x)f'(x) = \alpha f(x)\\
\end{align*}

Write $f'(x)= \frac{dy}{dx}$, and $f(x) = y$, now we have a separable
first order ordinary differential equation,
\[(1+x)\frac{dy}{dx}= \alpha y \implies \int \frac{dy}{y}= \int
  \frac{\alpha}{1+x} dx \implies \log y = \alpha\log (1+x) + K
  \implies y = e^Ke^{\alpha\log(1+x)} = C(1+x)^\alpha \]

If $K= 0\implies C = 1$, thus for $-1 < x < 1$,
\[(1+x)^\alpha = 1 + \sum_{n=1}^\infty \frac{\alpha(\alpha
    -1)\cdots(\alpha-n+1)}{n!} x^n\]

Now, for the other equality note by 8.18 (a) $0 < \alpha < \infty$ implies,
\begin{align*}
  \Gamma(n + \alpha)
  &= \Gamma(1 + n-1 + \alpha)\\
  &= (n-1 + \alpha) \Gamma(n-1 + \alpha)\\
  \Gamma(n-1 + \alpha)
  &= \Gamma(1 + n-2 + \alpha)\\
  &= (n-2 + \alpha) \Gamma(n-2 + \alpha)\\
  \Gamma(n-2 + \alpha)
  &= \Gamma(1 + n-3 + \alpha)\\
  &= (n-3 + \alpha) \Gamma(n-3 + \alpha)\\
  &\vdots\\
  \Gamma(n-(n-2) + \alpha)
  &= \Gamma(1 + 1 + \alpha)\\
  &= (1 + \alpha) \Gamma(1 + \alpha)\\
  \Gamma(n-(n-1) + \alpha)
  &= \Gamma(1  + \alpha)\\
  &= \alpha \Gamma(\alpha)
\end{align*}
Therefore $\Gamma(n+\alpha) = (n-1 + \alpha)(n-2 + \alpha)(n-3 +
\alpha)\cdots (2+\alpha)(1+\alpha)\alpha \Gamma(\alpha)$ is a
reasonable formula, which we want to prove by induction.

The base case is just 8.18 (a), now assume the formula above holds
for all $k\leq n$.
\[\Gamma(n+1 +\alpha) = \Gamma(1 + n+\alpha) =
  (n+\alpha)\Gamma(n+\alpha) = (n + \alpha)(n-1 + \alpha)(n-2 + \alpha)(n-3 +
  \alpha)\cdots (2+\alpha)(1+\alpha)\alpha \Gamma(\alpha)\]
The third equality holds by 8.18 (a) and the fourth by plugging in our
assumption for $n$. Thus the induction is complete and the formula
works for all natural numbers and $0<\alpha<\infty$. Then,
\[\frac{\Gamma(n+\alpha)}{\Gamma(\alpha)} = (n-1 + \alpha)(n-2 + \alpha)(n-3 +
  \alpha)\cdots (2+\alpha)(1+\alpha)\alpha\]

Note, \[-\alpha(-\alpha
    -1)\cdots(-\alpha -n + 2 )(-\alpha-n+1) = -(n-1+\alpha)\cdot
    -(n-2+\alpha)\cdots-(1+\alpha)-\alpha\]

There are $n$ terms in the product since you start at $(n-1+\alpha)$
and you end at $(n-n+\alpha)= \alpha$, so,
\[-\alpha(-\alpha
    -1)\cdots(-\alpha -n + 2 )(-\alpha-n+1) = (-1)^n(n-1 + \alpha)(n-2 + \alpha)(n-3 +
    \alpha)\cdots (2+\alpha)(1+\alpha)\alpha = (-1)^n\frac{\Gamma(n+\alpha)}{\Gamma(\alpha)}\]

Since $(1+x)^\alpha = 1 + \sum_{n=1}^\infty \frac{\alpha(\alpha
    -1)\cdots(\alpha-n+1)}{n!} x^n$, plugging in $-\alpha$ and $-x$
  for $|-x| = |x|<1$, we get,

  \[(1+(-x))^{-\alpha} = 1 + \sum_{n=1}^\infty \frac{-\alpha(-\alpha
    -1)\cdots(-\alpha-n+1)}{n!} (-x)^n = 1 + \sum_{n=1}^\infty
  \frac{(-1)^n\Gamma(n+\alpha)}{n!\Gamma(\alpha)} (-1)^nx^n\]

So, the $(-1)^n$ cancel each other and we have for $-1<x<1$ and
$\alpha > 0$,

\[(1-x)^{-\alpha} = 1 + \sum_{n=1}^\infty
  \frac{\Gamma(n+\alpha)}{n!\Gamma(\alpha)} x^n\]
\end{document}

%%% Local Variables:
%%% mode: latex
%%% TeX-master: t
%%% End:
