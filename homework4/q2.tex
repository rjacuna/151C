\documentclass{article}
\usepackage{fontspec}

% Used to embed Sage code in latex
\usepackage{sagetex}

% Math Environment
\usepackage{euler}        % Euler font
\usepackage{amsmath}      % Math macros
\usepackage{amssymb}      % Math symbols
\usepackage{unicode-math} % Unicode support


\usepackage[makeroom]{cancel} % Used to cancel terms in algebraic equations
\usepackage{ulem} % Different underline environments
\usepackage{polynom} %Polynomial long division

% Typesetting Rules
\setlength\parindent{0em}
\setlength\parskip{0.618em}
\usepackage[a4paper,lmargin=1in,rmargin=1in,tmargin=1in,bmargin=1in]{geometry}
\setmainfont[Mapping=tex-text]{Helvetica Neue LT Std 45 Light}

% Common Macros
\newcommand\N{\mathbb{N}}
\newcommand\Z{\mathbb{Z}}
\newcommand\Q{\mathbb{Q}}
\newcommand\R{\mathbb{R}}
\newcommand\C{\mathbb{C}}
\newcommand\A{\mathbb{A}}
\def\res{\mathop{\text{Res}}\limits}

% Color
\usepackage[dvipsnames]{xcolor}
\usepackage{pagecolor}
% \definecolor{DeepMossGreen}{HTML}{394820}
% \pagecolor{DeepMossGreen}
% \color{Goldenrod}

\begin{document}
The following simple computation yields a good approximation to
Stirling's formula.

For $m =1,2,\dots,$ define
\[f(x) = (m+1-x)\log m +(x-m)\log (m+1)\]
if $m \leq x \leq m+1$, and define
\[g(x) = \frac{x}{m} -1 + \log m\]
if $m-\frac{1}{2} \leq x\leq m+\frac{1}{2}$. Draw the graphs of $f$
and $g$.Note that $f(x)\leq \log(x)\leq g(x)$ if $x \geq 1$ and that
\[\int_1^n f(x) dx = \log(n!) -\frac{1}{2}\log n > -\frac{1}{8} +
  \int_{1}^n g(x) dx\text{.}\]
Integrate $\log x$ over $[1,n].$ Conclude that
\[\frac{7}{8}<\log (n!) -(n+\frac{1}{2})\log n +n < 1\]

for $n = 2,3,4,\dots.$ (\textit{Note} $\log \sqrt{2\pi} \sim
0.918\dots.)$ Thus
\[e^{7/8}<\frac{ n!}{(n/e)^n\sqrt{n}} < e\]

\begin{sagesilent}
  var('m,x')

  f(m,x) = (m +1 -x)*log(m) + (x-m)*log(m+1)
  g(m,x) = x/m -1 +log(m)
  f1(x) = (2-x)*log(1)+(x-1)*log(2)
  p = plot(f(1,x), (x,1,2), color='blue')
  p += plot(log(x),(x,1,7),color='orange')
  p += plot(g(1,x), (x,1,3/2), color='red')
  for n in range(2,7):
      p += plot(f(n,x),(x,n,n+1),color='blue')
      p += plot(g(n,x),(x,n-1/2,n+1/2),color='red')
  p +=plot(g(7,x), (x,7-1/2,7), color='red')
\end{sagesilent}

Note, Rudin is a fucking genius because the agreement is uncanny, if
I graph the interval $[0,100]$ my eyes can't tell the difference. So
$\color{blue}f\color{black}<\color{orange}\log\color{black}<\color{red}g$
for $x\geq1$. The
graphs correspond to the colors, and the range is $[1,7]$ for
emphasis. After $5$ the agreement is so good you can't even see
$\color{orange}\log\color{black}$ function.


$\sageplot{p}$


\begin{align*}
  \int_{1}^n f(x) dx
  &=\sum_{k=1}^n \int_{k}^{k+1} f(x) dx\\
  &=\sum_{k=1}^n \int_{k}^{k+1} (k+1-x)\log k +(x-k)\log (k+1) dx\\
  &=\sum_{k=1}^n \int_{k}^{k+1} (k+1)\log k -x\log k +x\log
    (k+1)-k\log (k+1) dx\\
  &=\sum_{k=1}^n [(k+1)\log k -k\log (k+1)][k+1-k]  +[\log
    (k+1)-\log k]\int_{k}^{k+1} x dx\\
  &=\sum_{k=1}^n [(k+1)\log k -k\log (k+1)][k+1-k]  +[\log
    (k+1)-\log k]\frac{1}{2}[(k+1)^2 - k^2]\\
  &=\sum_{k=1}^n [(k+1)\log k -k\log (k+1)][k+1-k]  +[\log
    (k+1)-\log k]\frac{1}{2}[k^2+2k + 1 - k^2]\\
  &=\sum_{k=1}^n [(k+1)\log k -k\log (k+1)][k+1-k]  +[\log
    (k+1)-\log k][k + \frac{1}{2}]\\
  &=\sum_{k=1}^n \log \left(\frac{k^{k+1}}{(k+1)^k}\right) + \log\left(  \left(
    \frac{k+1}{k}\right)^{k + \frac{1}{2}}\right) =\sum_{k=1}^n \log \left(\frac{k^{k+1}}{(k+1)^k}
    \frac{(k+1)^{k + \frac{1}{2}}}{k^{k + \frac{1}{2}}}\right)\\
  &= \sum_{k=1}^n \log \left(  k^{k+1-k-\frac{1}{2}}(k+1)^{k +
    \frac{1}{2}-k}\right)
    = \sum_{k=1}^n \log \left(
    k^{\frac{1}{2}}(k+1)^{\frac{1}{2}}\right) = \sum_{k=1}^n \log \left(
    k(k+1)\right)^{\frac{1}{2}}\\
  &= \sum_{k=1}^n \frac{1}{2} \log \left(  k(k+1)\right) =
    \frac{1}{2}\sum_{k=1}^n  \log k + \log (k+1)\\
  &= \frac{1}{2}\sum_{k=1}^n  \log k + \frac{1}{2}\sum_{k=1}^n \log
    (k+1)\\
    &= \frac{1}{2}\log(1) + \frac{1}{2}\sum_{k=2}^n  \log k + \frac{1}{2}\sum_{k=1}^{n-1} \log
    (k+1) + \frac{1}{2}\log (n+1)\\
  \text{ if } l = k+1 \text{ then } l \in [2,n] &\implies \frac{1}{2}\sum_{k=1}^{n-1} \log
                        (k+1) = \frac{1}{2}\sum_{l=2}^n \log l\\
\text{Thus} \int_{1}^n f(x) dx
  &= \frac{1}{2}\cancel{ \log(1)}+ \sum_{k=2}^n  \log k  +
    \frac{1}{2}\log (n+1)\\
  &= \log \prod_{k=2}^n k  + \frac{1}{2}\log (n+1)\\
  &= \log(n!)  + \frac{1}{2}\log (n+1)
\end{align*}
So, Rudin isn't that smart... I even checked by wolframalpha.com that
he's wrong. By quite a bit, but whatever.
\begin{align*}
  \int_{1}^n g(x) dx
  &= \int_{1}^{3/2} g(x) dx + \sum_{k=2}^{n-1} \int_{k-1/2}^{k+1/2} g(x)
    dx + \int_{n-1/2}^n g(x) dx\\
  &= \int_{1}^{3/2} x-1\, dx + \sum_{k=2}^{n-1} \int_{k-1/2}^{k+1/2}
    \frac{x}{k}-1 + \log k\,
    dx + \int_{n-1/2}^n \frac{x}{n}-1+\log n\, dx\\
  &=  \frac{x^2}{2}\bigg|_{1}^{3/2}-\frac{1/2} + \sum_{k=2}^{n-1} \int_{k-1/2}^{k+1/2}
    \frac{x}{k}-1 + \log k\,
    dx + \frac{x^2}{2n}\bigg|_{n}^{n-1/2} +[\log n
    -1]\left(n-n+\frac{1}{2}\right)\\
  &=  \frac{(3/2)^2-1}{2} -\frac{1}{2} + \sum_{k=2}^{n-1} \int_{k-1/2}^{k+1/2}
    \frac{x}{k}-1 + \log k\,
    dx + \frac{x^2}{2n}\bigg|_{n-1/2}^{n} +\frac{\log n
    -1}{2}\\
  &=  \frac{1}{8} + \frac{n^2 - (n-1/2)^2}{2n}-\frac{\log n
    -1}{2} + \sum_{k=2}^{n-1} \int_{k-1/2}^{k+1/2}
    \frac{x}{k}-1 + \log k\,
    dx  \\
  &=  \frac{1}{8} + \frac{n^2 - n^2 +n - 1/4}{2n}+\frac{\log n
    -1}{2} + \sum_{k=2}^{n-1} \int_{k-1/2}^{k+1/2}
    \frac{x}{k}-1 + \log k\,
    dx  \\
  &=  \frac{1}{8} + \frac{n}{2n} - \frac{1}{8n} +\frac{\log n
    -1}{2} + \sum_{k=2}^{n-1} \int_{k-1/2}^{k+1/2}
    \frac{x}{k}-1 + \log k\,
    dx  \\
  &=  \frac{1}{8} + \frac{1}{2} - \frac{1}{8n} +\frac{\log n
    -1}{2}  + \sum_{k=2}^{n-1}
    \frac{x^2}{2k}\bigg|_{k-1/2}^{k+1/2} + (k+1/2-(k-1/2))(\log k-1)
  \\
  &= \frac{1}{8} - \frac{1}{8n} +\frac{\log n}{2} + \sum_{k=2}^{n-1}
    \frac{(k+1/2)^2-(k-1/2)^2}{2k} + \log k-1 \\
  &= \frac{1}{8} - \frac{1}{8n} +\frac{\log n}{2} + \sum_{k=2}^{n-1}
    \frac{k^2+k +1/4 -(k^2-k +1/4)}{2k} + \log k-1 \\
  &= \frac{1}{8} - \frac{1}{8n} +\frac{\log n}{2} + \sum_{k=2}^{n-1}
    \frac{2k}{2k} + \log k-1 \\
&= \frac{1}{8} - \frac{1}{8n} +\frac{1}{2}\log n +
                                  \sum_{k=2}^{n-1}\log k \\
  &= \frac{1}{8} - \frac{1}{8n} -\frac{1}{2}\log n +
    \sum_{k=2}^{n-1}\log k +\log n \\
  &= \frac{1}{8} - \frac{1}{8n} -\frac{1}{2}\log n +
    \sum_{k=2}^{n}\log k \\
  &= \frac{1}{8} - \frac{1}{8n} -\frac{1}{2}\log n +
    \log(n!) \\
\end{align*}
Now,
\begin{align*}
  \int_{1}^n \log x\, dx &= x(\log x -1) \bigg|_{1}^n = n(\log n -1)
  -1(\log 1 -1) =  n\log n -n +1
\end{align*}

Now $f\leq\log\leq g \implies\int_{1}^n f dx \leq \int_{1}^n \log
dx\leq  \int g dx $. So,
\begin{align*}
  &\log(n!)  + \frac{1}{2}\log (n+1) \leq n\log n -n +1 \leq
    \frac{1}{8} - \frac{1}{8n} -\frac{1}{2}\log n + \log (n!)\\
  \implies &\frac{1}{2}\log (n+1) \leq n\log n -\log(n!) -n +1 \leq
             \frac{1}{8} - \frac{1}{8n} -\frac{1}{2}\log n\\
  \implies &\frac{1}{2}\log (n+1) -1 \leq n\log n -\log(n!) -n \leq
             -\frac{7}{8} - \frac{1}{8n} -\frac{1}{2}\log n \\
  \implies &1 > 1 -\frac{1}{2}\log (n+1) \geq \log(n!) +n -n\log n  \geq
             \frac{7}{8} + \frac{1}{8n} +\frac{1}{2}\log n > \frac{7}{8} \\
  \implies &e > n!e^{n -n\log n} > e^{\frac{7}{8}} \\
  \implies &e > \frac{n!}{(n/e)^n} > e^{\frac{7}{8}} \\
\end{align*}

Now this is the best I can do, however this formula is false, and
doesn't follow from $f$ and $g$. The counter example is ridiculously small...
\[\log(2!)  + \frac{1}{2}\log (2+1) \leq 2\log 2 -2 +1 \leq
    \frac{1}{8} - \frac{1}{16} -\frac{1}{2}\log 2 + \log (2!)\]
\[\implies 1.24 \dots \leq 0.38 \dots \leq 0.40\cdots\]

$g$ is fine. However, the counterexample contradicts the assumption
$f< \log$ in $[1,n]$.

Now, suppose there exists a piecewise function $h$ such that $h< \log$ in
$[1,n]$, and such that:
\[\int_{1}^n h dx = \log (n!) -\frac{1}{2}\log(n)\]
Then,
\begin{align*}
  &\log(n!) - \frac{1}{2}\log(n) \leq n\log n -n +1 \leq
    \frac{1}{8} - \frac{1}{8n} -\frac{1}{2}\log n + \log (n!)\\
  \implies&- \frac{1}{2}\log(n) \leq -\log (n!) + n\log n -n +1 \leq
            \frac{1}{8} - \frac{1}{8n} -\frac{1}{2}\log n\\
  \implies& 0 \leq -\log (n!) + \left(n+\frac{1}{2}\right)\log n -n +1 \leq
            \frac{1}{8} - \frac{1}{8n}\\
  \implies& -1 \leq -\log (n!) + \left(n+\frac{1}{2}\right)\log n -n \leq
            -\frac{7}{8} - \frac{1}{8n}\\
  \implies& 1 \geq \log (n!) - \left(n+\frac{1}{2}\right)\log n +n \geq
            \frac{7}{8} + \frac{1}{8n}\\
  \implies& 1 \geq \log (n!) - \left(n+\frac{1}{2}\right)\log n +n \geq
            \frac{7}{8} + \frac{1}{8n}>\frac{7}{8}\\
  \implies& e \geq n! n^{-\left(n+\frac{1}{2}\right)}e^n
            >e^{\frac{7}{8}}\\
  \implies& e \geq \frac{n!}{(n/e)^n\sqrt{n}} >e^{\frac{7}{8}}\\
\end{align*}
The preceding calculation was easy. However, how do I know that,
\[\log(n!) - \frac{1}{2}\log(n) \leq n\log n -n +1\text{ ?}\]

\begin{sagesilent}
  var('n')
  ih(n) = log(factorial(n)) - log(n)/2
  ilog(n) = n*log(n) -1*n +1
  def triplets (m):
      h = "n,\\log (n!) -\\frac{1}{2}\\log n, n\\log n -n + 1\\\\"
      s = [",".join([latex(k),latex(ih(k).n(digits=5)),latex(ilog(k).n(digits=5))])  for k in range(1,m+1)]
      return  h +"\\\\".join(s)
  def isit (m):
      s = [k  for k in range(1,m+1) if ((ilog(k).n(digits=27) -ih(k).n(digits=27)) < 0)]
      if s == []:
          return "\\text{yes}"
      else:
          return latex(s[0])
\end{sagesilent}
$\sagestr{triplets(173)}$

So, it's a reasonable assumption. Let's ask the computer if it holds
for $1\leq n \leq 20000$, else give me the first $n$ that fails. My
computer says $\sagestr{isit(20000)}$ it holds. However, for $1\leq
n\leq 100000$ my
computer was taking more than $5$ minutes which is longer than I'm willing
to wait.

Now, let's see if there's an $h$, that satisfies those properties. We
know,

\[\int_1^n f\, dx = \log(n!) +\frac{1}{2}\log(n+1)\quad \text{ and }
  \int_1^n h\, dx = \log(n!) -\frac{1}{2}\log(n)\]
\[\implies \int_1^n f\, dx -\int_1^n h\, dx = \log(n!)
  +\frac{1}{2}\log(n+1)  -\left(  \log(n!)
    -\frac{1}{2}\log n\right)\]
\[\implies \int_1^n f- h\, dx =
  \frac{1}{2}\log(n+1)  + \frac{1}{2}\log n\]
Now, I was going to give up there. However, wolframalpha.com came to
the rescue and found that,

\[\int_1^n \frac{1+2nx}{2x-2nx^2} dx = \frac{1}{2}\log(n+1)  + \frac{1}{2}\log n\]

Then \[f-h = \frac{1+2nx}{2x-2nx^2} \implies h(n,x) = f(x) -
  \frac{1+2nx}{2x-2nx^2}\]

If we could find a piecewise representation of $f-h$, for $m\leq x\leq
m+1$, then the definition of $h$ wouldn't depend on $n$.

However, we only need to define $h$ in the interval $[1,n]$. We need
to show $h < \log$ there. Actually, we also need to show $\log < g$
there. But, since Rudin didn't bother(and was wrong) I won't
either.

\end{document}

%%% Local Variables:
%%% mode: latex
%%% TeX-master: t
%%% End:
