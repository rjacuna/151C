\documentclass{article}
\usepackage{fontspec}

% Used to embed Sage code in latex
\usepackage{sagetex}

% Math Environment
\usepackage{euler}        % Euler font
\usepackage{amsmath}      % Math macros
\usepackage{amssymb}      % Math symbols
\usepackage{unicode-math} % Unicode support


\usepackage[makeroom]{cancel} % Used to cancel terms in algebraic equations
\usepackage{ulem} % Different underline environments
\usepackage{polynom} %Polynomial long division

% Typesetting Rules
\setlength\parindent{0em}
\setlength\parskip{0.618em}
\usepackage[a4paper,lmargin=1in,rmargin=1in,tmargin=1in,bmargin=1in]{geometry}
\setmainfont[Mapping=tex-text]{Helvetica Neue LT Std 45 Light}

% Common Macros
\newcommand\N{\mathbb{N}}
\newcommand\Z{\mathbb{Z}}
\newcommand\R{\mathbb{R}}
\newcommand\C{\mathbb{C}}
\newcommand\A{\mathbb{A}}
\def\res{\mathop{\text{Res}}\limits}

% Color
\usepackage[dvipsnames]{xcolor}
\usepackage{pagecolor}
% \definecolor{DeepMossGreen}{HTML}{394820}
% \pagecolor{DeepMossGreen}
% \color{Goldenrod}

\begin{document}

\begin{sagesilent}
  from matplotlib import rc
  rc('text', usetex=True)
  rc('text.latex', preamble=r'\usepackage{amssymb}')
  p = plot(1, (-1.618,1.618))
  p += circle((-1.618,0),0.01) + plot(0,(-pi + -1.618,- 1.618))
  p += circle((1.618,0),0.01) + plot(0,(1.618, pi + 1.618))
  t = text(r"$\delta$", (1.618, -0.2)) + text(r"$-\delta$", (-1.618, -0.2)) + p
  t.axes_color((0.8,0.8,0.8))
  t.axes_range(xmin= -pi/2 + -1.618, xmax = pi/2+1.618,ymin=None, ymax=None)
  t.axes_width(0.001)
\end{sagesilent}

The following is the subset of the plot of $f$ in $[-\pi,\pi].$

$\sageplot{t}$

(a)

Following formula (62)
\[c_m = \frac{1}{2\pi} \int_{-\pi}^\pi f(x) e^{-imx} dx\]

Since $f$ is $1$ for $|x|< \delta$ and  $0$ for $\delta < |x|\leq \pi$, we can see that,
\begin{align*}
  m\neq 0 \implies c_m &= \frac{1}{2\pi} \int_{-\delta}^\delta e^{-imx} dx\\
      &=  \frac{1}{2\pi} -\frac{e^{-imx}}{im}|_{-\delta}^\delta\\
      &=\frac{1}{2\pi} -\left(\frac{e^{-im\delta}}{im} -
        \frac{e^{im\delta}}{im}\right) \\
      &=\frac{1}{2\pi} \left(\frac{e^{im\delta}-
        e^{-im\delta}}{im}\right) \\
      &=\frac{1}{m\pi} \left(\frac{e^{im\delta}-
        e^{-im\delta}}{2i}\right) \\
                       &=\frac{1}{m\pi} \sin(m\delta)\\
  m=0 \implies c_0 &= \frac{1}{2\pi} \int_{-\delta}^\delta e^{0} dx  \\
                       &=\frac{\delta-(-\delta)}{2\pi} = \frac{\delta}{\pi}
\end{align*}

\[c_{-m} =\frac{\sin(-m\delta)}{-m\pi}  = \frac{-\sin(m\delta)}{-m\pi}
  =\frac{\sin(m\delta)}{m\pi} = \overline{c_m} \]

So, $f$ is real by the remark in page 186. Notice, $\cos(x) = \frac{e^{ix}+e^{-ix}}{2}\implies 2\cos(inx) =
e^{inx}+e^{-inx}$

Then we can see that if we change $m$ to $n$,

\[f(x) = \sum_{-\infty}^\infty c_n e^{inx} = c_0
  +\sum_{n=1}^\infty c_n e^{inx} + \sum_{n=1}^\infty c_{-n} e^{i-nx}  =
  c_0 +\sum_{n=1}^\infty c_n (e^{inx} + e^{i-nx}) = c_0
  +\sum_{n=1}^\infty c_n 2 \cos(nx)  \]

So $b_n = 0$ for $n\in \N$ and $a_n = 2c_n$, for $(n = 0,1, 2,
\dots)$.\footnote{This is why Boyce and DiPrima wrote $a_0/2$ in their
  definition of Fourier series.}
That is $a_0 =\frac{2\delta}{\pi}$, and $a_n =
\frac{2\sin(n\delta)}{n\pi}$ for $n\in \N$. Finally

\[f(x) = \frac{\delta}{\pi} +\sum_{n=1}^\infty \frac{2\sin(n\delta)}{n\pi}\cos(nx)   \]

\newpage
(b)
\[\forall t\in [-\delta,\delta]\quad  |f(0+t) -f(0)| = |1-1| = 0 \leq |t|\]

Thus by 8.14 $\lim_{n\rightarrow \infty}S_n(f;0) = f(0)$ , which means

\begin{align*}
  \frac{\delta}{\pi} +\sum_{n=0}^\infty   \frac{2\sin(n\delta)}{n\pi}\cos(0) = f(0)
  &\implies \frac{\delta}{\pi} +\sum_{n=0}^\infty
                                          \frac{2\sin(n\delta)}{n\pi}
    = 1 \\
  &\implies \frac{2}{\pi} \sum_{n=0}^\infty
                                          \frac{\sin(n\delta)}{n}
    = 1 - \frac{\delta}{\pi}\\
  &\implies \sum_{n=0}^\infty
                                          \frac{\sin(n\delta)}{n}
    = \frac{\pi}{2} - \frac{\delta}{2}\\
  &\implies \sum_{n=0}^\infty
                                          \frac{\sin(n\delta)}{n}
    = \frac{\pi-\delta}{2}
\end{align*}

(c)
Formula (84) gives us what we need,

\[\frac{1}{2\pi} \int_{-\pi}^\pi |f(x)|^2 dx = \sum_{-\infty}^\infty |c_n|^2\]

\begin{align*}
  \frac{1}{2\pi} \int_{-\pi}^\pi |f(x)|^2 dx
  &= \frac{1}{2\pi} \int_{-\delta}^\delta |1|^2 dx\\
  &=    \frac{\delta-(-\delta)}{2\pi}\\
  &= \frac{\delta}{\pi}\\
  &= \frac{\delta^2}{\pi^2} + 2\sum_{n=1}^\infty
    \frac{\sin^2(n\delta)}{n^2\pi^2}\\
  &= c_0^2 + 2\sum_{n=1}^\infty |c_n|^2\\
  &= c_0^2 + \sum_{n=1}^\infty |c_n|^2+\sum_{n=1}^\infty |c_n|^2\\
  &= c_0^2 + \sum_{n=1}^\infty |c_{-n}|^2+\sum_{n=1}^\infty |c_n|^2\\
  &= \sum_{-\infty}^\infty |c_n|^2\\
\frac{\delta}{\pi}
  = \frac{\delta^2}{\pi^2} + 2\sum_{n=1}^\infty
  \frac{\sin^2(n\delta)}{n^2\pi^2}
  &\implies \delta\pi  =\delta^2 + 2\sum_{n=1}^\infty
    \frac{\sin^2(n\delta)}{n^2}\\
  &\implies \pi  =\delta + 2\sum_{n=1}^\infty
    \frac{\sin^2(n\delta)}{n^2\delta}\\
  &\implies \sum_{n=1}^\infty
    \frac{\sin^2(n\delta)}{n^2\delta} = \frac{\pi-\delta}{2}\\
\end{align*}
\end{document}
%%% Local Variables:
%%% mode: latex
%%% TeX-master: t
%%% End:
