\documentclass{article}
\usepackage{fontspec}

% Used to embed Sage code in latex
\usepackage{sagetex}

% Math Environment
\usepackage{euler}        % Euler font
\usepackage{amsmath}      % Math macros
\usepackage{amssymb}      % Math symbols
\usepackage{unicode-math} % Unicode support


\usepackage[makeroom]{cancel} % Used to cancel terms in algebraic equations
\usepackage{ulem} % Different underline environments
\usepackage{polynom} %Polynomial long division

% Typesetting Rules
\setlength\parindent{0em}
\setlength\parskip{0.618em}
\usepackage[a4paper,lmargin=1in,rmargin=1in,tmargin=1in,bmargin=1in]{geometry}
\setmainfont[Mapping=tex-text]{Helvetica Neue LT Std 45 Light}

% Common Macros
\newcommand\N{\mathbb{N}}
\newcommand\Z{\mathbb{Z}}
\newcommand\R{\mathbb{R}}
\newcommand\C{\mathbb{C}}
\newcommand\A{\mathbb{A}}
\def\res{\mathop{\text{Res}}\limits}

% Color
\usepackage[dvipsnames]{xcolor}
\usepackage{pagecolor}
% \definecolor{DeepMossGreen}{HTML}{394820}
% \pagecolor{DeepMossGreen}
% \color{Goldenrod}

\begin{document}
Let $f(x) = x^3 - \sin^2(x)\tan(x)$ and $g(x) = 2x^2 - \sin^2 x - x\tan x$

Since as $x\rightarrow \pi/2$ both $f$ and $g$ go to $-\infty$, we
only need to show if $\exists\quad 0<x<\pi/2$ such that $f$ or $g$ are
positive.
\begin{sageblock}
  var('x')
  f = x^3 -1*sin(x)^2*tan(x)
  g = 2*x^2 -1*sin(x)*sin(x)-1*x*tan(x)
  def fderivatives (h,m):
      return ",".join(["f^{(%d)}(0) =" % n + latex(h.derivative(x,n).subs(x=0)) for n in range(m+1)])
  def gderivatives (h,m):
      return ",".join(["g^{(%d)}(0) =" % n + latex(h.derivative(x,n).subs(x=0)) for n in range(m+1)])
\end{sageblock}

The computation above shows that,
$\sagestr{fderivatives(f,4)}$ and

$f^{(5)}(x)=$ $\sagestr{latex(f.derivative(x,5).factor())}$

$= -(120 \, \sin\left(x\right)^{2} \tan\left(x\right)^{6}$
$+ 240 \, \cos\left(x\right) \sin\left(x\right)
\tan\left(x\right)^{5} + 120 \, \cos\left(x\right)^{2}
\tan\left(x\right)^{4}$
$ + 120 \, \sin\left(x\right)^{2} \tan\left(x\right)^{4}$

$ + 240 \, \cos\left(x\right) \sin\left(x\right) \tan\left(x\right)^{3}$
$ + 120 \, \cos\left(x\right)^{2} \tan\left(x\right)^{2} + 16 \,
\sin\left(x\right)^{2} \tan\left(x\right)^{2} + 32 \,
\cos\left(x\right) \sin\left(x\right) \tan\left(x\right) $
$+ 16 \, \sin\left(x\right)^{2} )$

Since $\sin(x)\cos(x)\tan(x) = \sin^2(x)$, we have,


$f^{(5)}(x)$$= -(120 \, \sin\left(x\right)^{2} \tan\left(x\right)^{6}$
$+ 240 \,  \sin^2\left(x\right)
\tan\left(x\right)^{4} + 120 \, \cos\left(x\right)^{2}
\tan\left(x\right)^{4}$
$ + 120 \, \sin\left(x\right)^{2} \tan\left(x\right)^{4}$

$ + 240 \, \sin^2\left(x\right) \tan\left(x\right)^{2}$
$ + 120 \, \cos\left(x\right)^{2} \tan\left(x\right)^{2} + 16 \,
\sin\left(x\right)^{2} \tan\left(x\right)^{2} + 48 \,
\sin^2\left(x\right)) $

All of the terms inside the parentheses are positive, so the fifth
derivative is negative always. So, $f$ never changes sign.


Also, $\sagestr{gderivatives(g, 3)}$, and

$g^{(4)} =$ $\sagestr{latex(g.derivative(x,4).factor())}$

$= -(24 \, x \tan\left(x\right)^{5} + 40 \, x \tan\left(x\right)^{3} + 24 \, \tan\left(x\right)^{4} - 8 \, \cos\left(x\right)^{2} + 8 \, \sin\left(x\right)^{2} + 16 \, x \tan\left(x\right) + 32 \, \tan\left(x\right)^{2} + 8
)$

Since $x\tan x$ is positive in $(0,\pi/2)$ and $8-8\cos^2(x)=
8\sin^2(x)$ it follows $g^{(4)}$ is also negative. So, $g$ never changes
sign. So, $g$ is negative.
\end{document}

%%% Local Variables:
%%% mode: latex
%%% TeX-master: t
%%% End:
