\documentclass{article}
\usepackage{fontspec}

% Used to embed Sage code in latex
%\usepackage{sagetex}

% Math Environment
\usepackage{euler}        % Euler font
\usepackage{amsmath}      % Math macros
\usepackage{amssymb}      % Math symbols
\usepackage{unicode-math} % Unicode support


\usepackage[makeroom]{cancel} % Used to cancel terms in algebraic equations
\usepackage{ulem} % Different underline environments
\usepackage{polynom} %Polynomial long division

% Typesetting Rules
\setlength\parindent{0em}
\setlength\parskip{0.618em}
\usepackage[a4paper,lmargin=1in,rmargin=1in,tmargin=1in,bmargin=1in]{geometry}
\setmainfont[Mapping=tex-text]{Helvetica Neue LT Std 45 Light}

% Common Macros
\newcommand\N{\mathbb{N}}
\newcommand\Z{\mathbb{Z}}
\newcommand\Q{\mathbb{Q}}
\newcommand\R{\mathbb{R}}
\newcommand\C{\mathbb{C}}
\newcommand\A{\mathbb{A}}
\def\res{\mathop{\text{Res}}\limits}

% Color
\usepackage[dvipsnames]{xcolor}
\usepackage{pagecolor}
% \definecolor{DeepMossGreen}{HTML}{394820}
% \pagecolor{DeepMossGreen}
% \color{Goldenrod}

\begin{document}

Use the hint, exists $\phi$ on $[a,b]$ with $\phi' = \gamma'/\gamma$
and $\phi(a) = 0$. That $\phi$ must be the one on $FTC$ 1, i.e.

$\phi(x) =\int_a^x \frac{\gamma'(t)}{\gamma(t)} dt$

Since $\gamma\exp(-\phi)$ is constant, it follows $\exp(\phi(b)) =
1\implies \phi(b) = -2\pi in$. So $\text{Ind}(\gamma)$ is some
integer. Now compute,
\[\frac{1}{2\pi i}\int_0^{2\pi} \frac{ine^{int}}{e^{int}} dt  =
  \frac{n}{2\pi }\int_0^{2\pi} dt = n\]

So, if $\gamma = e^{int}$ then  Ind($\gamma$) $= n$.


\end{document}

%%% Local Variables:
%%% mode: latex
%%% TeX-master: t
%%% End:
