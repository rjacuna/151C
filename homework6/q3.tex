\documentclass{article}
\usepackage{fontspec}

% Used to embed Sage code in latex
%\usepackage{sagetex}

% Physics Environment
\usepackage{physics}

% Math Environment
\usepackage{euler}        % Euler font
\usepackage{amsmath}      % Math macros
\usepackage{amssymb}      % Math symbols
\usepackage{unicode-math} % Unicode support


\usepackage[makeroom]{cancel} % Used to cancel terms in algebraic equations
\usepackage{ulem} % Different underline environments
\usepackage{polynom} %Polynomial long division

% Typesetting Rules
\setlength\parindent{0em}
\setlength\parskip{0.618em}
\usepackage[a4paper,lmargin=1in,rmargin=1in,tmargin=1in,bmargin=1in]{geometry}
\setmainfont[Mapping=tex-text]{Helvetica Neue LT Std 45 Light}

% Common Macros
\newcommand\N{\mathbb{N}}
\newcommand\Z{\mathbb{Z}}
\newcommand\Q{\mathbb{Q}}
\newcommand\R{\mathbb{R}}
\newcommand\C{\mathbb{C}}
\newcommand\A{\mathbb{A}}
\def\res{\mathop{\text{Res}}\limits}

% Color
\usepackage[dvipsnames]{xcolor}
\usepackage{pagecolor}
% \definecolor{DeepMossGreen}{HTML}{394820}
% \pagecolor{DeepMossGreen}
% \color{Goldenrod}

\begin{document}

If $f(0,0) = 0$ and \[ f(x,y) = \frac{xy}{x^2+y^2},\quad \text{ if
  }(x,y)\neq (0,0),\]
Prove that $(D_1 f)(x,y)$ and $(D_2 f)(x,y)$ exist at every point of
$R^2$, although $f$ is not continuous at $(0,0).$

\uwave{pf.}

In $\R^2$,
\[(x,y) \neq (0,0) \iff x^2+y^2 \neq 0\]
Thus $\forall (x,y)\in \R^2: (x,y)\neq (0,0)$ gives us that,
\begin{equation}
  (D_1 f)(x,y) =  \frac{y^3 -yx^2}{(x^2+y^2)^2}
\end{equation}
and
\begin{equation}
  (D_2 f)(x,y) =  \frac{x^3 -xy^2}{(x^2+y^2)^2}
\end{equation}
both exist, since their common denominator is never zero.


If $(x,y) = (0,0)$, formula (25) gives,
\begin{equation*}
(D_1 f)(0,0) = \lim_{t \rightarrow 0}
               \frac{f(0+t,0)-f(0,0)}{t}
  = \lim_{t \rightarrow 0}
    \frac{f(t,0)-0}{t}
  = \lim_{t \rightarrow 0}
    \frac{\frac{t\cdot 0}{t^2+ 0^2}}{t}
  = 0
\end{equation*}
and by symmetry of $f$,
\begin{equation*}
(D_2 f)(0,0) = \lim_{t \rightarrow 0}
               \frac{f(0,0+t)-f(0,0)}{t} = 0
\end{equation*}

So,\[\forall (x,y)\in \R^2,\,\exists (D_1 f)(x,y)\text{ and }(D_2 f)(x,y)\]

Along the diagonal $x = y$,
\[\lim_{(x,y) \rightarrow (0,0)} f(x,y) = \lim_{x \rightarrow 0}
  \frac{x^2}{x^2 + x^2} =   \frac{1}{2}\]
And along the $x$-axis,
\[\lim_{(x,y) \rightarrow (0,0)} f(x,y) = \lim_{x \rightarrow 0}
  \frac{x\cdot 0}{x^2 + 0^2} =  0\]

It follows that the limit doesn't exist, so $f$ is not continuous at $(0,0)\quad \blacksquare$
\end{document}



%%% Local Variables:
%%% mode: latex
%%% TeX-master: t
%%% End:
