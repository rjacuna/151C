\documentclass{article}
\usepackage{fontspec}

% Used to embed Sage code in latex
%\usepackage{sagetex}

% Physics Environment
\usepackage{physics}

% Math Environment
\usepackage{euler}        % Euler font
\usepackage{amsmath}      % Math macros
\usepackage{amssymb}      % Math symbols
\usepackage{unicode-math} % Unicode support


\usepackage[makeroom]{cancel} % Used to cancel terms in algebraic equations
\usepackage{ulem} % Different underline environments
\usepackage{polynom} %Polynomial long division

% Typesetting Rules
\setlength\parindent{0em}
\setlength\parskip{0.618em}
\usepackage[a4paper,lmargin=1in,rmargin=1in,tmargin=1in,bmargin=1in]{geometry}
\setmainfont[Mapping=tex-text]{Helvetica Neue LT Std 45 Light}

% Common Macros
\newcommand\N{\mathbb{N}}
\newcommand\Z{\mathbb{Z}}
\newcommand\Q{\mathbb{Q}}
\newcommand\R{\mathbb{R}}
\newcommand\C{\mathbb{C}}
\newcommand\A{\mathbb{A}}
\def\res{\mathop{\text{Res}}\limits}

% Color
\usepackage[dvipsnames]{xcolor}
\usepackage{pagecolor}
% \definecolor{DeepMossGreen}{HTML}{394820}
% \pagecolor{DeepMossGreen}
% \color{Goldenrod}

\begin{document}

Suppose that $f$ is a differentiable real function in an open set
$E\subset R^n$, and that $f$ has a local maximum at a point $\vb{x}\in
E$. Prove that $f'(\vb{x}) = 0$

\uwave{slu.}

Fix $\vb{y}\in \R^n$,

Let $\gamma: [-1,1]\longrightarrow \R^m; t\mapsto \vb{x} + t\vb{y}$

Let $\phi: [-1,1]\longrightarrow \R; \phi =  f \circ \gamma$

Since $\gamma$ is differentiable, $(-1,1)\subset [-1,1]$ is open, and $f$ maps
$\gamma(-1,1)$ into $\R$, it follows by Theorem
9.15 that $\phi$ is differentiable on $(-1,1)$. And formula (22) holds
 for $-1<t<1$,
\begin{equation}\phi'(t) = f'(\gamma(t))\gamma'(t)\end{equation}

Since, $\vb{x}$ is a local maximum of $f$. It follows that $0$
is a  local maximum of $\phi$, since $\phi(0) = f(\vb{x})$.

Since $\phi$ has  local maximum at $t=0$, it follows from Theorem 5.8
that $\phi'(0)= 0$, then
\begin{equation} f'(\gamma(0))\gamma'(0) = f'(\vb{x}) \vb{y} = 0\end{equation}

Since $\vb{y}$ was an arbitrary non zero vector it follows that,
\[ f'(\vb{x}) = 0\quad \blacksquare\]

\end{document}



%%% Local Variables:
%%% mode: latex
%%% TeX-master: t
%%% End:
