\documentclass{article}
\usepackage{fontspec}

% Used to embed Sage code in latex
%\usepackage{sagetex}


% Math Environment
\usepackage{euler}        % Euler font
\usepackage{amsmath}      % Math macros
\usepackage{amssymb}      % Math symbols
\usepackage{unicode-math} % Unicode support

% Physics Environment
\usepackage{physics}


\usepackage[makeroom]{cancel} % Used to cancel terms in algebraic equations
\usepackage{ulem} % Different underline environments
\usepackage{polynom} %Polynomial long division

% Typesetting Rules
\setlength\parindent{0em}
\setlength\parskip{0.618em}
\usepackage[a4paper,lmargin=1in,rmargin=1in,tmargin=1in,bmargin=1in]{geometry}
\setmainfont[Mapping=tex-text]{Helvetica Neue LT Std 45 Light}

% Common Macros
\newcommand\N{\mathbb{N}}
\newcommand\Z{\mathbb{Z}}
\newcommand\Q{\mathbb{Q}}
\newcommand\R{\mathbb{R}}
\newcommand\C{\mathbb{C}}
\newcommand\A{\mathbb{A}}
\def\res{\mathop{\text{Res}}\limits}

% Color
\usepackage[dvipsnames]{xcolor}
\usepackage{pagecolor}
% \definecolor{DeepMossGreen}{HTML}{394820}
% \pagecolor{DeepMossGreen}
% \color{Goldenrod}

\usepackage{manfnt}

\begin{document}

Decide whether the following is true or false and prove your
conclusion.

Statement: Let $f:\R^m \rightarrow \R$  be a continuous function such
that for every $x\in\R^m$ and every unit vector $e\in \R^m$, the
directional derivative of $f$ at $x$ in the direction of $e$
exists. Then $f$ is differentiable.

\uwave{slu.}

It is false!

Consider function,
\[f(x,y) = \begin{cases} 0\hspace{1.91em},\text{ if } (x,y) = (0,0)\\
    \frac{x^3}{x^2+y^2},\text{ else}\end{cases}\]

Since the limit,\[\lim_{(x,y)\rightarrow (0,0)} \frac{x^3}{x^2+y^2} =
  0\]
$f$ is continuous, as $(0,0)$ is the only point where it could fail to
be continuous.

Let $\vb{u} = (u_1,u_2)\in \R^2$ be a unit vector. If $\vb{x} = (x,y)\in \R^2: \vb{x}\neq
(0,0)$, then formula $(40)$, gives us,
\begin{align*}
  (D_u f)(x,y) &= (D_1 f)(x,y)u_1 + (D_2 f)(x,y)u_2\\
               &= \frac{x^4+3x^2y^2}{(x^2+y^2)^2}u_1 - \frac{2x^3y}{(x^2+y^2)^2}u_2
\end{align*}
Since $(x,y)\neq (0,0) \implies x^2+y^2\neq 0$, it follows that for
all $(x,y)\neq (0,0)$ the directional derivative exists in the
direction of every unit vector $\vb{u}$.

At $(0,0)$, equation (39) gives us,
\begin{align*}
  (D_u f)(0,0) &= \lim_{t\rightarrow 0} \frac{f(tu_1,tu_2) -
                 f(0,0)}{t}\\
               &= \lim_{t\rightarrow 0} \frac{\frac{(tu_1)^3}{(tu_1)^2+(tu_2)^2} - 0
                 }{t}\\
               &= \lim_{t\rightarrow 0} \frac{u_1^3}{u_1^2+u_2^2}\\
               &= u_1^3
\end{align*}

So, the directional derivatives of $f$ along every unit vector
$\vb{u}\in\R^2$ exist for all points $\vb{x}\in \R^2$.

Let $\{\vb{e}_1,\vb{e}_2\}$ be the standard basis of $\R^2$.

If, the derivative exists at $(0,0),$ then \[f'(0,0) = [(D_{\vb{e}_1}
  f)(0,0),(D_{\vb{e}_2} f)(0,0)] = [1,0]\]

Then,  $[1,0],$ must satisfy equation (14) at $(0,0)$, but if $\vb{h}=
(h_1,h_2)$

\begin{align*}
  \lim_{\vb{h}\rightarrow 0} \frac{|f(\vb{h})-f(0)
  -[1,0]\vb{h}|}{|\vb{h}|}  &= \lim_{\vb{h}\rightarrow 0} \frac{|h_1^3
        -h_1^3 -h_1h_2^2|}{(h_1^2+h_2^2)^{3/2}}\\
          &= \lim_{\vb{h}\rightarrow 0}
            \frac{|h_1|h_2^2}{(h_1^2+h_2^2)^{3/2}}\\
            &= \lim_{h_1=h_2 = h \rightarrow 0}
              \frac{h^3}{(2h^2)^{3/2}}\\
              &=
                \frac{1}{2^{3/2}}\neq 0
\end{align*}

So, $f$ is not differentiable at $(0,0)$. So, $f$ contradicts the
proposition \textdbend

\end{document}



%%% Local Variables:
%%% mode: latex
%%% TeX-master: t
%%% End:
