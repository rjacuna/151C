\documentclass{article}
\usepackage{fontspec}

% Used to embed Sage code in latex
%\usepackage{sagetex}

% Math Environment
\usepackage{euler}        % Euler font
\usepackage{amsmath}      % Math macros
\usepackage{amssymb}      % Math symbols
\usepackage{unicode-math} % Unicode support


\usepackage[makeroom]{cancel} % Used to cancel terms in algebraic equations
\usepackage{ulem} % Different underline environments
\usepackage{polynom} %Polynomial long division

% Typesetting Rules
\setlength\parindent{0em}
\setlength\parskip{0.618em}
\usepackage[a4paper,lmargin=1in,rmargin=1in,tmargin=1in,bmargin=1in]{geometry}
\setmainfont[Mapping=tex-text]{Helvetica Neue LT Std 45 Light}

% Common Macros
\newcommand\N{\mathbb{N}}
\newcommand\Z{\mathbb{Z}}
\newcommand\R{\mathbb{R}}
\newcommand\C{\mathbb{C}}
\newcommand\A{\mathbb{A}}
\def\res{\mathop{\text{Res}}\limits}

% Color
\usepackage[dvipsnames]{xcolor}
\usepackage{pagecolor}
 \definecolor{DeepMossGreen}{HTML}{394820}
\pagecolor{DeepMossGreen}
\color{Goldenrod}

\begin{document}

\begin{center} \section*{$\zeta(2) = \frac{\pi^2}{6}$} \end{center}
Consider $f(x) = x$ for $x$ in $[-\pi,\pi]$,
\begin{alignat*}{2}
  C_0 &= \frac{1}{2\pi}\int_{-\pi}^\pi x dx = 0& &\text{ Since $x$ is
    an odd function}\\
  C_n &= \frac{1}{2\pi} \int_{-\pi}^\pi xe^{inx}dx &&\quad n \neq 0 \\
  &= \frac{1}{2\pi}\left( x\frac{e^{inx}}{in}\bigg|_{-\pi}^\pi
    -\int_{-\pi}^\pi\frac{e^{inx}}{in} dx \right) &&\quad u = x\quad dv = e^{inx}dx\\
  &&&\quad du = dx\quad v = \frac{e^{inx}}{in}\\
  &= \frac{1}{2\pi}\left( \pi\frac{e^{in\pi}}{in} - (-\pi)\frac{e^{-in\pi}}{in}
    -\frac{e^{inx}}{(in)^2}\bigg|_{-\pi}^\pi \right) &&\\
  &= \frac{1}{2\pi}\left( \pi\frac{e^{in\pi}}{in} - (-\pi)\frac{e^{-in\pi}}{in}
    -\left(  \frac{e^{in\pi}}{(in)^2}
      -\frac{e^{-in\pi}}{(in)^2}\right) \right) &&\\
  &= \frac{1}{2\pi}\left( \pi\frac{e^{in\pi}}{in} +\pi\frac{e^{-in\pi}}{in}
    -\frac{e^{in\pi}}{(in)^2}
      +\frac{e^{-in\pi}}{(in)^2} \right) &&\,e^{\pm in\pi}= \cos(n\pi)
    \pm i \sin(n\pi) = \cos(n\pi) = (-1)^n\\
  &= \frac{1}{2\pi}\left( \pi\frac{(-1)^n}{in} +\pi\frac{(-1)^n}{in}
    -\frac{(-1)^n}{(in)^2}
    +\frac{(-1)^n}{(in)^2} \right) &&= \frac{1}{2\pi}\left( \pi\frac{(-1)^n}{in} +\pi\frac{(-1)^n}{in}
    -\cancel{\frac{(-1)^n}{(in)^2}}
    +\cancel{\frac{(-1)^n}{(in)^2}} \right)\\
  &= \frac{1}{2\pi}\left( 2\pi\frac{(-1)^n}{in}\right) =
  \frac{(-1)^n}{in} = \frac{-i(-1)^n}{n}&&\\
  &= \frac{i(-1)^{n+1}}{n}
\end{alignat*}
\begin{align*}
  \frac{1}{2\pi}\int_{-\pi}^\pi |x|^2 dx
  &=\frac{1}{2\pi}\int_{-\pi}^0 (-x)^2 dx +\frac{1}{2\pi}\int_{0}^\pi
    x^2 dx\\
  &= \frac{1}{2\pi}\left(  \frac{x^3}{3}\bigg|_{-\pi}^0 +
    \frac{x^3}{3}\bigg|_{0}^\pi \right)\\
  &=\frac{1}{2\pi}\left( -\frac{(-\pi)^3}{3} + \frac{\pi^3}{3}
    \right)  \\
  &= \frac{1}{2\pi}\left(\frac{2\pi^3}{3} \right)\\
  &= \frac{\pi^2}{3} = 2\sum_{n=1}^\infty \frac{1}{n^2}\\
  &= \sum_{n=1}^\infty \frac{1}{(-n)^2}+ \sum_{n=1}^\infty \frac{1}{n^2}\\
  &= \sum_{-\infty}^\infty \frac{1}{n^2}\\
  &= \sum_{-\infty}^\infty \frac{|i|^2|(-1)^{n+1}|^2}{n^2}\\
  &= \sum_{-\infty}^\infty \bigg|\frac{i(-1)^{n+1}}{n}\bigg|^2
\end{align*}
Thus, \[\zeta(2) = \sum_{n=1}^\infty \frac{1}{n^2} = \frac{\pi^2}{6}\]
\end{document}
%%% Local Variables:
%%% mode: latex
%%% TeX-master: t
%%% End:
