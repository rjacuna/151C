\documentclass{article}
\usepackage{fontspec}

% Used to embed Sage code in latex
%\usepackage{sagetex}

% Math Environment
\usepackage{euler}        % Euler font
\usepackage{amsmath}      % Math macros
\usepackage{amssymb}      % Math symbols
\usepackage{unicode-math} % Unicode support


\usepackage[makeroom]{cancel} % Used to cancel terms in algebraic equations
\usepackage{ulem} % Different underline environments
\usepackage{polynom} %Polynomial long division

% Typesetting Rules
\setlength\parindent{0em}
\setlength\parskip{0.618em}
\usepackage[a4paper,lmargin=1in,rmargin=1in,tmargin=1in,bmargin=1in]{geometry}
\setmainfont[Mapping=tex-text]{Helvetica Neue LT Std 45 Light}

% Common Macros
\newcommand\N{\mathbb{N}}
\newcommand\Z{\mathbb{Z}}
\newcommand\Q{\mathbb{Q}}
\newcommand\R{\mathbb{R}}
\newcommand\C{\mathbb{C}}
\newcommand\A{\mathbb{A}}
\def\res{\mathop{\text{Res}}\limits}

% Color
\usepackage[dvipsnames]{xcolor}
\usepackage{pagecolor}
% \definecolor{DeepMossGreen}{HTML}{394820}
% \pagecolor{DeepMossGreen}
% \color{Goldenrod}

\begin{document}

$|\sin 0x| = 0 =  0|\sin x|$ and  $|\sin 1x| = 1|\sin x|$

Assume,\[\exists n\in \N: \forall k\in\N: k\leq n \implies |\sin kx|
  \leq  k|\sin x|\]

Then,
\begin{align*} |\sin (n+1)x|  = |\sin (nx + x)| &= |\sin nx\cos x + \cos(nx)\sin x|\\
  &\leq |\sin nx\cos x| + |\cos(nx)\sin x| = |\sin nx||\cos x| +
                               |\cos(nx)||\sin x|\\
  \forall n \in \N, |\cos nx|\leq 1 &\implies |\sin(n+1)x|\leq |\sin nx| +
                                      |\sin x| \\
  |\sin nx|
  \leq  n|\sin x|&\implies |\sin (n+1)x| \leq n|\sin x| +
                   |\sin x| = (n+1)|\sin x|\end{align*}

Therefore the inequality $|\sin nx|\leq n|\sin x|$ is valid for every
natural number and zero.

Note, \[0.001 \approx |\sin 0.001| > 0.001|\sin(1)| \approx 0.0008417\]
\end{document}

%%% Local Variables:
%%% mode: latex
%%% TeX-master: t
%%% End:
