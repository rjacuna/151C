\documentclass{article}
\usepackage{fontspec}

% Used to embed Sage code in latex
%\usepackage{sagetex}

% Math Environment
\usepackage{euler}        % Euler font
\usepackage{amsmath}      % Math macros
\usepackage{amssymb}      % Math symbols
\usepackage{unicode-math} % Unicode support


\usepackage[makeroom]{cancel} % Used to cancel terms in algebraic equations
\usepackage{ulem} % Different underline environments
\usepackage{polynom} %Polynomial long division

% Typesetting Rules
\setlength\parindent{0em}
\setlength\parskip{0.618em}
\usepackage[a4paper,lmargin=1in,rmargin=1in,tmargin=1in,bmargin=1in]{geometry}
\setmainfont[Mapping=tex-text]{Helvetica Neue LT Std 45 Light}

% Common Macros
\newcommand\N{\mathbb{N}}
\newcommand\Z{\mathbb{Z}}
\newcommand\R{\mathbb{R}}
\newcommand\C{\mathbb{C}}
\newcommand\A{\mathbb{A}}
\def\res{\mathop{\text{Res}}\limits}

% Color
\usepackage[dvipsnames]{xcolor}
\usepackage{pagecolor}
% \definecolor{DeepMossGreen}{HTML}{394820}
% \pagecolor{DeepMossGreen}
% \color{Goldenrod}

\begin{document}

(a)

$b^x = e^{x\log b} (b > 0)$ by formula (43) in p. 181.

So, $(b^x - 1)' = (e^{x\log b} - 1)' =e^{x\log b}(x \log b)' +(-1)' = e^{x\log
  b} \log b  + 0= e^{x\log
  b} \log b$

Since $x' = 1$, we have\[\lim_{x\rightarrow 0} e^{x\log
    b} \log b = \lim_{x\rightarrow 0} \sum_{k = 0}^\infty \frac{\left( x \log b
    \right)^k}{k!} = \sum_{k = 0}^\infty
  \frac{\lim_{x\rightarrow 0} \left( x \log b
    \right)^k}{k!} = \sum_{k = 0}^\infty
  \frac{ \left( \lim_{x\rightarrow 0} x \log b
    \right)^k}{k!} = \sum_{k = 0}^\infty
  \frac{ 0^k}{k!} =  e^0 \log b
  = \log b\]
 As $x\rightarrow 0$ \[b^x-1 \longrightarrow b^0-1 = e^{0\log b} -1 =
   1-1 = 0\]
 So by L'Hospital's rule,
 \[\lim_{x\rightarrow 0}\frac{b^x-1}{x} = \log b\]

 (b)

 Since, $\log(1+x)' = \frac{1}{1+x}$ and $x' = 1$. $\lim_{x\rightarrow
   0} = \frac{1}{1+0} =1$.
 As $x\rightarrow 0$,\[\log(1+x) \longrightarrow \log(1+0) = \log 1 = 0\]

 So by L'Hospital's rule,
 \[\lim_{x\rightarrow 0} \frac{\log(1+x)}{x} = 1\]

 (c)

 We know from Theorem $3.31$ that, \[\lim_{n\rightarrow \infty}\left(
     1 + \frac{1}{n} \right)^n = e\]

 Put $x = \frac{1}{n}$, then $n = 1/x$, and  \[x \longrightarrow 0 \iff
   n\longrightarrow \infty \]
 So, \[\lim_{n\rightarrow \infty}\left(
     1 + \frac{1}{n} \right)^n = e = \lim_{x\rightarrow 0} \left( 1 +
     x\right)^{1/x}\]

(d)
\begin{align*}
   \lim_{n\rightarrow \infty}\left( 1 + \frac{x}{n} \right)^n
   &= \lim_{n\rightarrow \infty} \sum_{k=0}^n\begin{pmatrix}n\\k\end{pmatrix}
   1^{n-k}\frac{x^{k}}{n^{k}}\\
   &= \lim_{n\rightarrow \infty} \sum_{k=0}^n\frac{n!}{k!(n-k)!} \frac{x^{k}}{n^{k}}\\
   &= \lim_{n\rightarrow \infty} \sum_{k=0}^n\frac{x^{k}}{k!}
     \frac{n!}{n^{k}(n-k)!}\\
   &= \lim_{n\rightarrow \infty} \sum_{k=0}^n\frac{x^{k}}{k!}
     \frac{(n-1)(n-2)\cdots(n-(k+1)}{n^{k-1}}\\
  &= \sum_{k=0}^n\frac{x^{k}}{k!} \lim_{n\rightarrow \infty}
    \frac{n^{k-1}+ P(n)}{n^{k-1}}\text{ where }\deg P(n) < k-1\\
     &= \sum_{k=0}^n\frac{x^{k}}{k!} \\
  &= e^x
\end{align*}

Note, we could've used (d) to prove (c) instead of invoking $3.31$
\end{document}
%%% Local Variables:
%%% mode: latex
%%% TeX-master: t
%%% End:
