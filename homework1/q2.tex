\documentclass{article}
\usepackage{fontspec}

% Used to embed Sage code in latex
%\usepackage{sagetex}

% Math Environment
\usepackage{euler}        % Euler font
\usepackage{amsmath}      % Math macros
\usepackage{amssymb}      % Math symbols
\usepackage{unicode-math} % Unicode support


\usepackage[makeroom]{cancel} % Used to cancel terms in algebraic equations
\usepackage{ulem} % Different underline environments
\usepackage{polynom} %Polynomial long division

% Typesetting Rules
\setlength\parindent{0em}
\setlength\parskip{0.618em}
\usepackage[a4paper,lmargin=1in,rmargin=1in,tmargin=1in,bmargin=1in]{geometry}
\setmainfont[Mapping=tex-text]{Helvetica Neue LT Std 45 Light}

% Common Macros
\newcommand\N{\mathbb{N}}
\newcommand\Z{\mathbb{Z}}
\newcommand\R{\mathbb{R}}
\newcommand\C{\mathbb{C}}
\newcommand\A{\mathbb{A}}
\def\res{\mathop{\text{Res}}\limits}

% Color
\usepackage[dvipsnames]{xcolor}
\usepackage{pagecolor}
% \definecolor{DeepMossGreen}{HTML}{394820}
% \pagecolor{DeepMossGreen}
% \color{Goldenrod}

\begin{document}

(a)
\[(e -(1+x)^{1/x})'  = -((1+x)^{1/x})'\]
\[(1+x)^{1/x} = e^{\frac{1}{x} \log{(1+x)}} \implies \log
  (1+x)^{1/x} = \frac{1}{x} \log{(1+x)} \implies  \left(\log
    (1+x)^{1/x} \right)'= \left(  \frac{1}{x} \log{(1+x)}\right)'\]
\begin{align*}
  \left(\log (1+x)^{1/x} \right)'
  &= \frac{1}{(1+x)^{1/x}} \left( (1+x)^{1/x} \right)'\\
  &= -\frac{1}{x^2}\log(1+x) +\frac{1}{x}\frac{1}{(1+x)}\\
  &= \left(  \frac{1}{x} \log{(1+x)}\right)'
\end{align*}
So,\[ \frac{1}{(1+x)^{1/x}} \left( (1+x)^{1/x} \right)'
  = -\frac{1}{x^2}\log(1+x) +\frac{1}{x}\frac{1}{(1+x)} \implies
  \left( (1+x)^{1/x} \right)'  = (1+x)^{1/x}\left(-\frac{1}{x^2}\log(1+x)
    +\frac{1}{x}\frac{1}{(1+x)}\right)\]
\[\lim_{x\rightarrow 0} (1+x)^{1/x}\left(-\frac{1}{x^2}\log(1+x)
    +\frac{1}{x}\frac{1}{(1+x)}\right) = \lim_{x\rightarrow 0}
  (1+x)^{1/x} \lim_{x\rightarrow 0}\left(-\frac{1}{x^2}\log(1+x)
    +\frac{1}{x}\frac{1}{(1+x)}\right)\]
Note,
\[-\frac{1}{x^2}\log(1+x)
    +\frac{1}{x}\frac{1}{(1+x)} = \frac{x -(1+x)\log(1+x) }
    {x^2(1+x)} = \frac{x -(1+x)\log(1+x) }
    {x^2+x^3}\]
Taking derivatives gives,
  \[(x -(1+x)\log(1+x))' = 1 - \log(1+x) -1 = -\log(1+x)\]
and,
\[(x^2+x^3)' = 2x+3x^2\]
Taking derivatives again,
\[(-\log(1+x))' = -\frac{1}{1+x}\]
and,
\[(2x+3x^2)' = 2+6x\text{ .}\]

Now, \[\lim_{x\rightarrow 0} \frac{-\frac{1}{1+x}}{2+6x} =
  -\frac{1}{2}\]

Since as $x\longrightarrow 0$,\[-\log(1+x) \rightarrow 0\text{ and }
  2x+3x^2 \longrightarrow 0\]
Also, \[x -(1+x)\log(1+x) \longrightarrow 0 \text{ and }
  x^2(1+x)\longrightarrow 0\]

So, \[\lim_{x\rightarrow 0} -\frac{1}{x^2}\log(1+x)
  +\frac{1}{x}\frac{1}{(1+x)} = - \frac{1}{2}\]
And since \[\lim_{x\rightarrow 0} (1+x)^{1/x} = e\]
We have \[\lim_{x\rightarrow 0} -((1+x)^{1/x})' = -\lim_{x\rightarrow
    0} ((1+x)^{1/x})' = -1\cdot e\cdot(-\frac{1}{2}) = \frac{e}{2}\]
As $x\rightarrow 0$,\[e-(1+x)^{1/x} \rightarrow 0\]
So finally by the applications of L'Hospital's rule above,
\[\lim_{x\rightarrow 0} \frac{e-(1+x)^{1/x}}{x} = \frac{e}{2}\]
\newpage
(b)

Note, $\frac{n}{\log n}[n^{1/n} - 1] = \frac{[n^{1/n} -1]}{\frac{\log n}{n}}$

\[
  n^{1/n} = e^{\frac{1}{n}\log n}
  \implies \log  n^{1/n} = \frac{1}{n}\log n
  \implies (\log  n^{1/n})' = \frac{1}{n^{1/n}}\left(n^{1/n}\right)' = \left(\frac{1}{n}\log n\right)'\\
\]
Thus,
\[ (n^{1/n}- 1)' = (n^{1/n})'= n^{1/n}\left(\frac{\log n}{n}\right)'\]
So,\[\lim_{n\rightarrow \infty} \frac{\left(  n^{1/n}
      -1\right)'}{\left(\frac{\log n}{n}\right)'} = \lim_{n\rightarrow
  \infty} \frac{n^{1/n}\left(\frac{\log n}{n}\right)'}{
  \left(\frac{\log n}{n}\right)'} = \lim_{n\rightarrow \infty}
n^{1/n} = 1\]

Now, $n= n^1$, and $1\neq -1$. So, by formula (45) in
p. 181, \[\lim_{n\rightarrow \infty} \frac{\log{n}}{n} = 0\]
Note, as $n\rightarrow \infty$, \[n^{1/n} -1 \rightarrow 0\]
So, by L'Hospital's rule, \[\lim_{n\rightarrow \infty} \frac{n}{\log
    n}[n^{1/n}-1] = 1\]
(c)

Compute the power series for $\tan x$,
\begin{align*}
  (\tan x)^{(0)} = \tan x  &\implies  a_0 = 0 \\
  (\tan x)^{(1)} = \sec^2 x  &\implies a_1  = 1 \\
  (\tan x)^{(2)} = 2\sec^2 x \tan x &\implies a_2  = 0 \\
  (\tan x)^{(3)} = 6\sec^4 x -4\sec^2 x &\implies a_3  = \frac{2}{3!}
                                          = \frac{1}{3}\\
  (\tan x)^{(4)} = 8\sec^2(x) \tan(x) (2 \sec^2(x) + \tan^2(x))
                           &\implies a_4 =0\\
  (\tan x)^{(5)} = 8 (2 \sec^6(x) + 11 \sec^4(x) \tan^2(x) + 2
  \sec^2(x) \tan^4(x)) &\implies a_5 = \frac{16}{5!} = \frac{2}{15}
\end{align*}
Then, \[\tan x = x +\frac{1}{3}x^3 + \frac{2}{15}x^5 + \cdots \implies \tan{x} -x  =
  \frac{1}{3}x^3+  \frac{2}{15}x^5 +\cdots\]
  Now,\[\cos x = (1-\frac{1}{2}x^2+ \frac{1}{4!}x^4 +\cdots) \implies
  x(1-\cos x) = x(1-1 + \frac{1}{2}x^2 -\frac{1}{4!}x^4\cdots) = \frac{1}{2}x^3
  -\frac{1}{4!}x^5 + \cdots\]
So, \[\frac{\tan{x} -x}{x(1-\cos x)} = \frac{\frac{1}{3}x^3 +  \frac{2}{15}x^5 +\cdots}{\frac{1}{2}x^3
  -\frac{x^5}{4!} + \cdots}  = \frac{1/x^3}{1/x^3}\frac{\frac{1}{3}x^3 +  \frac{2}{15}x^5 +\cdots}{\frac{1}{2}x^3
  -\frac{x^5}{4!} + \cdots}   = \frac{\frac{1}{3} +  \frac{2}{15}x^2 +\cdots}{\frac{1}{2}
  -\frac{x^2}{4!} + \cdots}  \]
Then,
\[\lim_{x\rightarrow 0} \frac{\tan{x} -x}{x(1-\cos x)} =\lim_{x\rightarrow 0}
  \frac{\frac{1}{3} +  \frac{2}{15}x^2 +\cdots}{\frac{1}{2}
    -\frac{x^2}{4!} + \cdots}  = \frac{2}{3} \]

(d)
\[\sin x = x -\frac{1}{6}x^3+\frac{1}{120}x^5 - \cdots \implies x - \sin x =
  \frac{1}{6}x^3-\frac{1}{120}x^5 + \cdots\]
So, \[\frac{x-\sin x}{\tan x -x} =
  \frac{\frac{1}{6}x^3-\frac{1}{120}x^5 + \cdots}{\frac{1}{3}x^3 +
    \frac{2}{15}x^5 +\cdots} = \frac{1/x^3}{1/x^3}\frac{\frac{1}{6}x^3-\frac{1}{120}x^5 + \cdots}{\frac{1}{3}x^3 +
    \frac{2}{15}x^5 +\cdots}  = \frac{\frac{1}{6}-\frac{1}{120}x^2 + \cdots}{\frac{1}{3} +
    \frac{2}{15}x^2 +\cdots} \]
Then,\[\lim_{x\rightarrow 0} \frac{x-\sin x}{\tan x - x} =
  \lim_{x\rightarrow 0} \frac{\frac{1}{6}-\frac{1}{120}x^2 + \cdots}{\frac{1}{3} +
    \frac{2}{15}x^2 +\cdots} = \frac{1}{2}\]

\end{document}

%%% Local Variables:
%%% mode: latex
%%% TeX-master: t
%%% End:
