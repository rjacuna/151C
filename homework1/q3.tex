\documentclass{article}
\usepackage{fontspec}

% Used to embed Sage code in latex
%\usepackage{sagetex}

% Math Environment
\usepackage{euler}        % Euler font
\usepackage{amsmath}      % Math macros
\usepackage{amssymb}      % Math symbols
\usepackage{unicode-math} % Unicode support


\usepackage[makeroom]{cancel} % Used to cancel terms in algebraic equations
\usepackage{ulem} % Different underline environments
\usepackage{polynom} %Polynomial long division

% Typesetting Rules
\setlength\parindent{0em}
\setlength\parskip{0.618em}
\usepackage[a4paper,lmargin=1in,rmargin=1in,tmargin=1in,bmargin=1in]{geometry}
\setmainfont[Mapping=tex-text]{Helvetica Neue LT Std 45 Light}

% Common Macros
\newcommand\N{\mathbb{N}}
\newcommand\Z{\mathbb{Z}}
\newcommand\Q{\mathbb{Q}}
\newcommand\R{\mathbb{R}}
\newcommand\C{\mathbb{C}}
\newcommand\A{\mathbb{A}}
\def\res{\mathop{\text{Res}}\limits}

% Color
\usepackage[dvipsnames]{xcolor}
\usepackage{pagecolor}
% \definecolor{DeepMossGreen}{HTML}{394820}
% \pagecolor{DeepMossGreen}
% \color{Goldenrod}

\begin{document}
(a) Assume $f$ is differentiable.
\[f(x)f(y) = f(x+y) \implies f(0)f(1) = f(0+1) = f(1) \implies f(0) =
  1\]
\[f'(x) = \lim_{h\rightarrow 0} \frac{f(x+h) - f(x)}{h} =
  \lim_{h\rightarrow 0} \frac{f(x)f(h) - f(x)}{h}  = f(x) \lim_{h\rightarrow 0} \frac{f(h) - 1}{h}\]
\[f(0) = 1 \implies \lim_{h\rightarrow 0} \frac{f(h) - 1}{h} =
  \lim_{h\rightarrow 0} \frac{f(h) - f(0)}{h} = \lim_{h\rightarrow 0}
  \frac{f(0+h) - f(0)}{h} = f'(0)\]
\[f'(0) = \lim_{h\rightarrow 0} \frac{f(h) - 1}{h} \implies f'(x) =
  f'(0)f(x)\]

$g(x) = e^{cx}$, when $f'(x) = c$, satisfies the same
properties.Therefore,
\[\left(\frac{f}{g}\right)' = \frac{cfg -cfg}{g^2} = 0 \implies
  \frac{f}{g} = M \in \R \implies \forall x\in \R,  f(x) = M g(x)\]

$g(0) = 1 = f(0) \implies M = 1$, therefore $f(x)= e^{cx}.$

(b) Assume $f$ is continuous,

Let  $x= y = 1$ in $f(x)f(y) = f(x+y) \implies f(2) = f(1)^2$.
Assume $f(k) = f(1)^k$ for all $k \in \N: k\leq n$,
\[f(n)f(1) = f(1)^nf(1)  \implies f(n+1) = f(1)^{n+1}\]

So, by induction $\forall n\in \N, f(n)=f(1)^n$.

\[f(x)f(y) = f(x+y) \implies f(-x)f(x) = f(-x+x) = f(0) = 1 \implies
  \forall x \in \R, f(-x) = \frac{1}{f(x)}\]
\[ f(-1) = \frac{1}{f(1)} \implies  \forall n\in\N, f(-n) =\frac{1}{f(n)} =
  \frac{1}{f(1)^n}\]

So, \[\forall n\in \Z, f(n) = f(1)^n\]

From $f(x)f(y) = f(x+y)$ an induction argument and extension to $\Z$ like above shows, \[
  \forall p\in \Z, x\in \R, f\left(p x\right) =
  f\left(x\sum_{k=1}^{p} 1\right) = f\left(x
  \right)^p \quad (1)\]
So,
\[f(px) = f(x)^p \implies \log f(px) = p \log f(x) \implies
  \frac{1}{p} \log f(px) = \log f(x) \implies f(x) = e^{\frac{1}{p}
    \log f(px)} = f(px)^{1/p}\]

Let $x = \frac{1}{p}$ then, \[f(x) = f(px)^{1/p} \implies
  f\left(\frac{1}{p}\right) = f\left(p\frac{1}{p}\right)^{1/p} = f(1)^{1/p} \]

So, \[\forall p\in \Z, f\left(\frac{1}{p}\right) = f(1)^{1/p}\]

Thus by equation $(1)$,
\[\forall r \in \Q,\quad  f\left(r \right) = f(1)^{r}\]

Since, $\Q$ is dense in $\R$ and $f$ agrees with $g (x) = e^{cx}$ at all
rational numbers, $f = g$ by a previous exercise.
\end{document}

%%% Local Variables:
%%% mode: latex
%%% TeX-master: t
%%% End:
