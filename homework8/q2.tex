\documentclass{article}
\usepackage{fontspec}

% Used to embed Sage code in latex
%\usepackage{sagetex}


% Math Environment
\usepackage{euler}        % Euler font
\usepackage{amsmath}      % Math macros
\usepackage{amssymb}      % Math symbols
\usepackage{unicode-math} % Unicode support

% Physics Environment
\usepackage{physics}


\usepackage[makeroom]{cancel} % Used to cancel terms in algebraic equations
\usepackage{ulem} % Different underline environments
\usepackage{polynom} %Polynomial long division

% Typesetting Rules
\setlength\parindent{0em}
\setlength\parskip{0.618em}
\usepackage[a4paper,lmargin=1in,rmargin=1in,tmargin=1in,bmargin=1in]{geometry}
\setmainfont[Mapping=tex-text]{Helvetica Neue LT Std 45 Light}

% Common Macros
\newcommand\N{\mathbb{N}}
\newcommand\Z{\mathbb{Z}}
\newcommand\Q{\mathbb{Q}}
\newcommand\R{\mathbb{R}}
\newcommand\C{\mathbb{C}}
\newcommand\A{\mathbb{A}}
\def\res{\mathop{\text{Res}}\limits}

% Color
\usepackage[dvipsnames]{xcolor}
\usepackage{pagecolor}
% \definecolor{DeepMossGreen}{HTML}{394820}
% \pagecolor{DeepMossGreen}
% \color{Goldenrod}

\usepackage{graphicx}

\begin{document}

Suppose that $f:\R^m\rightarrow\R^n$ is continuously
differentiable. Prove that there exists some non-negative integer $r$
and a non-empty open set $E\subset \R^m$ such that
\[\text{rant }f'(\vb{x})= r\text{ for all }\vb{x}\in E.\]

\uwave{pf.}



Proposition 2 of (Lewis, 2009), establishes that the rank of a linear
map is lower semicontinuous.

Proposition 2 means in particular that set $\{A\in L(\R^m,\R^n)|\, \text{rank }(A)>
r\}$ is open for all $r\geq 0$.

Let $r\geq 0$, then

$\forall \vb{x}\in \R^m,\, 0\leq \text{ rank }f'(\vb{x}) \leq n \implies \exists
\vb{x_0}\in \R^m: \forall \vb{x}\in\R^m: \text{ rank }(f'(\vb{x}))\leq \text{ rank
}(f'(\vb{x_0})) = r$

So, since $f$ is continuously differentiable, we have that,
\[\exists \delta>0:\forall \varepsilon >0,\, |x-x_0|<\delta \implies
  \|f'(\vb{x})-f'(\vb{x_0})\| < \varepsilon\]
Then the lower semicontinuity of rank tells us that,
\[\|f'(\vb{x})-f'(\vb{x_0})\| < \varepsilon \implies \text{rank
  }(f'(\vb{x_0}))\leq \text{rank
  }(f'(\vb{x}))\]
But since $\vb{x_0}$ is a global maximum of $f'$, it follows that,
\[\forall\vb{x}\in \R^m:|\vb{x}-\vb{x_0}|<\delta \implies \text{rank
  }(f'(\vb{x})) = r\quad \blacksquare\]

\subsection{Bibliography}
Lewis, Andrew. \textit{Semicontinuity of rank and nullity and some
  consequences}. Online, 2009.

\hspace{7em}URL https://mast.queensu.ca/~andrew/notes/pdf/2009a.pdf.
\end{document}



%%% Local Variables:
%%% mode: latex
%%% TeX-master: t
%%% End:
